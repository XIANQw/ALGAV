\documentclass[14px]{article}
\usepackage{xeCJK}
\usepackage[frenchb]{babel}
\usepackage[T1]{fontenc}
\usepackage[utf8]{inputenc}
\usepackage{textcomp}
\usepackage{amssymb}
\usepackage[ruled,longend]{algorithm2e}
\usepackage{amsmath}
\usepackage{latexsym}
\usepackage{fancyhdr}
\usepackage{geometry}
\usepackage{setspace}
\renewcommand{\baselinestretch}{1.2}

\begin{document}
\setlength{\parindent}{0pt}
\begin{titlepage}
	\thispagestyle{empty}
	\raggedleft 
	\rule{1pt}{\textheight} % Vertical line
	\hspace{0.05\textwidth} % Whitespace between the vertical line and title page text
	\parbox[b]{0.75\textwidth}{ % Paragraph box for holding the title page text, adjust the width to move the title page left or right on the page
		
		{\Huge\bfseries Calculer le rectange \\minimum convexe}\\[2\baselineskip] % Title
		{\Large\textsc{Qiwei XIAN}} % Author name, lower case for consistent small caps
		
		\vspace{0.5\textheight} % Whitespace between the title block and the publisher

		{\noindent Sorbonne Université}\\[\baselineskip] % Publisher and logo
	}
\end{titlepage}
\clearpage

\tableofcontents
\thispagestyle{empty}
\clearpage

\pagestyle{fancy}	
\lhead{Calculer le rectangle minimum convexe}	
\rhead{\thepage}
\fancyfoot{}

\section{Introduction}
Donner une ensemble $\mathcal{P}$ de n points dans $\mathbb{R}$, le rectangle minimum convexe est le rectangle minimum qui entourne tous les points de $\mathcal{P}$.\\\\
Calculer le rectangle minimum convexe est un des problèmes classiques de la géométrie algorithmique. Pour résoudre ce problème, un majorant de $\mathcal{O}(n^2)$ est donné par l'algorithm recherche exhaustive. Il existe aussi les autre meilleurs algorithmes comme \textbf{l'algorithme Shamos}, le premier fois que Michael Shamos l'a utilisé afin de calculer la diamère d’un polygon convexe en temps $\mathcal{O}(n)$ en 1978. En plus \textbf{l'algorithme Toussaint}, Godfried Toussaint a résolu beaucoup de problèmes de la géométrie algorithmique en utilisant la phase "\textbf{rotating caliper}". Dans mon projet, je cherche à calculer le rectangle minimum convexe en utilisant l'algorithme Toussaint.\\\\
Néanmoins il faut trouver l'enveloppe convexe de $\mathcal{P}$ avant d'utiliser ces algorithmes. Pour faire ce précalcul, il y a plusieurs choix possibles. Par exemple le \textbf{parcours de Graham}, il nous permet de calculer un enveloppe convexe en temps $\mathcal{O}(n\log n)$. Un autre algorithme la \textbf{marche de Jarvis}, il a aussi une excellente complexité en $\mathcal{O}(nh)$ où h représente le nombre de sommets de l'enveloppe convexe. \\\\
Dans ce projet, je choisis la marche de Jarvis afin de précalculer l'enveloppe convexe de $\mathcal{P}$. Dès que obtenir l'envelloppe, je vais utiliser rotating caliper pour chercher les qutre points de l'enveloppe qui se trouve aussi dans le rectangle. Après je peux calculer les sommets de rectangle en utilisant ces quatre points. La complexité théorique est $\mathcal{O}(n + r)$ où r est la complexité de la marche de Jarvis.
\clearpage

\section{Définitions, notations, stuctures de donnés utilisée}

\begin{enumerate}
	\item $\overrightarrow{ab}$ est représente la vecteur depuis $a$ vers $b$, $|\overrightarrow{ab}|$ est le scalaire de $\overrightarrow{ab}$.
	\item $\overrightarrow{ab}\cdot\overrightarrow{cd}$ dénote le produit scalaire entre $\vec{ab}$ et $\vec{cd}$.
	\item $\overrightarrow{ab}\times\overrightarrow{cd}$ dénote le produit vectoriel entre $\overrightarrow{ab}$ et $\overrightarrow{cd}$.
\end{enumerate}

On dénote l'ensembles de points par $\mathcal{P}$, le nombre de points de $\mathcal{P}$ par $n$.\\
Une enveloppe convexe $\mathrm{E}$ de $\mathcal{P}$ est un sous-ensemble $\mathrm{E}$ de $\mathcal{P}$ en ordre du sens horaire tel que le polygon composé par tous les points de $\mathrm{E}$ peut entourner tous les points de $\mathcal{P}$.\\
Un rectangle convexe $Rec$ est une liste de qutre points. $RecMin$ est le rectangle minimun convexe. $AR$ est une ensemble de quatre arêtes de $Rec$.\\

\textbf{Lemma 1} if $Rec$ est $RecMin$, alors $\exists \mathcal{B}$, $\mathcal{B} \in \mathrm{E}$ tel que $\forall b, b \in \mathcal{B},\exists c, ar, c \in Rec, ar\in AR$, $\overrightarrow{bc}$ sont colinéaire avec $ar$.

\textbf{Lemma 2} il existe une arête $\alpha$ de $RecMin$, $\alpha$ passe une arête du polygon minimum convexe.


\section{Présentation de l'algorithm de Toussaint}
Afin de calculer les sommets A, B, C, D de rectangle, on a besoin d'abord de trouver $\alpha$ qui est colinéaire avec une arête de rectangle et les points $A^{\prime}$,  $B^{\prime}$, $C^{\prime}$, $D^{\prime}$ de $\mathcal{B}$ qui est sur les arêtes de $RecMin$.

\subsection{Etape 1 calculer $A^{\prime}$,  $B^{\prime}$, $C^{\prime}$, $D^{\prime}$}
Pour $A^{\prime}$, on peut énumérer chaque points $i$ de $\mathrm{E}$, $A^{\prime} = E_{i}$,
et $\alpha = \overrightarrow{E_{i+1}E_{i}}$ \\

Pour $B^{\prime}$, il est le plus ouest point tel que $|\alpha \cdot \overrightarrow{B^{\prime}A^{\prime}}|$ est maximum, il exprime $\angle \alpha\overrightarrow{B^{\prime}A^{\prime}}$ est le plus grand. Comme $\mathrm{E}$ est en ordre du sens horaire, on énumère chaque point $E_{r}$ de $\mathrm{E}$, $|\alpha\cdot\overrightarrow{E_{r}A^{\prime}}| \leqslant |\alpha\cdot\overrightarrow{E_{r+1}A^{\prime}}|$, si une fois $|\alpha\cdot\overrightarrow{E_{r}A^{\prime}}| > |\alpha\cdot\overrightarrow{E_{r+1}A^{\prime}}|$, alors $B^{\prime} = E_{r}$\\

Pour $C^{\prime}$, il est le point antipodal de $A^{\prime}$, la distance entre $A^{\prime}$ et $C^{\prime}$ est plus grande que $A^{\prime}$ et les autres points, donc $|\alpha\times\overrightarrow{E_{u}A^{\prime}}|$ est maximum. Comme $\mathrm{E}$ est en ordre du sens horaire, si on énumère chaque point $E_{u}$ de $\mathrm{E}$, au début $|\alpha\times\overrightarrow{E_{u}A^{\prime}}| \leqslant |\alpha\times\overrightarrow{E_{u+1}A^{\prime}}|$, si une fois $|\alpha\times\overrightarrow{E_{u}A^{\prime}}| > |\alpha\times\overrightarrow{E_{u+1}A^{\prime}}|$, alors $C^{\prime}$ = $E_{u}$\\

Le dernier point $D^{\prime}$ est le plus est point, $\angle \alpha D^{\prime}$ est le plus petit. On énumère chaque point $E_{l}$ de $\mathrm{E}$, comme $\mathrm{E}$ est en ordre du sens horaire, au début $|\alpha\cdot\overrightarrow{E_{l}A^{\prime}}| \geq|\alpha\cdot\overrightarrow{E_{l+1}A^{\prime}}|$, si une fois $|\alpha\cdot\overrightarrow{E_{l}A^{\prime}}| < |\alpha\cdot\overrightarrow{E_{l+1}A^{\prime}}|$, alors $D^{\prime}$ = $E_{l}$\\

\subsection{Etape 2 calculer l'aire $\mathcal{S}$ de rectangle}
Pour la hauteur de rectangle $\mathrm{h}$, on suppose que le point $C^{\prime}_{\alpha}$ est le projecteur de $C^{\prime}$ sur $\alpha$, $\mathrm{h}= |\overrightarrow{C^{\prime}_{\alpha}C^{\prime}}|$  
$\mathcal{S}_{\text{parallélogramme}}$ = $|\vec{\alpha}\times\overrightarrow{C^{\prime}A^{\prime}}|$. \\
$\mathrm{h} =  \frac{\mathcal{S}_{\text{parallélogramme}}}{|\vec{\alpha}|}$\\

Pour la largeur de rectangle $\mathrm{w}$, on suppose que $B^{\prime}_{\alpha}$ est le projecteur de $B^{\prime}$ sur $\alpha$ et $D^{\prime}_{\alpha}$ est le projecteur de $D^{\prime}$ sur $\alpha$, $\mathrm{w} = |\overrightarrow{B^{\prime}_{\alpha}D^{\prime}_{\alpha}}|$\\
$\overrightarrow{B^{\prime}_{\alpha}A^{\prime}}$ = $\frac{\alpha\cdot\overrightarrow{B^{\prime}A^{\prime}}}{|\alpha|}$\\
$\overrightarrow{D^{\prime}_{\alpha}A^{\prime}}$ = $\frac{\alpha\cdot\overrightarrow{D^{\prime}A^{\prime}}}{|\alpha|}$\\
$|\overrightarrow{B^{\prime}_{\alpha}D^{\prime}_{\alpha}}| = |\overrightarrow{D^{\prime}_{\alpha}A^{\prime}} - \overrightarrow{B^{\prime}_{\alpha}A^{\prime}}|$\\
Donc, $\mathcal{S} = h \times w$. Si $\mathcal{S} < \mathcal{S}_{RecMin}$, on va continue à l'étape 3 afin de construire le nouveau $RecMin$ sinon on essaie le $A^{\prime}$ suivant.\\ 


\subsection{Etape 3 calculer $A$, $B$, $C$, $D$}
Dès que'on obtient $A^{\prime}$,  $B^{\prime}$, $C^{\prime}$, $D^{\prime}$, $\alpha$, $\overrightarrow{B^{\prime}_{\alpha}A^{\prime}}$, $\overrightarrow{B^{\prime}_{\alpha}D^{\prime}_{\alpha}}$,
$\overrightarrow{C^{\prime}_{\alpha}C^{\prime}}$, on peut facilement calculer $A$, $B$, $C$, $D$.\\
$A$ est le point $B^{\prime}_{\alpha}$, $B^{\prime}_{\alpha} = A^{\prime} - \alpha\cdot\frac{\overrightarrow{B^{\prime}_{\alpha}A^{\prime}}}{|\alpha|}$\\
$B$ est le point suivant de $A$ au sens de horaire, $B =  B^{\prime}_{\alpha} +  \overrightarrow{B^{\prime}_{\alpha}B^{\prime}}\cdot\frac {\overrightarrow{C^{\prime}_{\alpha}C^{\prime}}}{|\overrightarrow{B^{\prime}_{\alpha}B^{\prime}}|}$\\
$C$ est le point suivant de $B$ au sens de horaire, $C = B + \overrightarrow{B^{\prime}_{\alpha}A^{\prime}}\cdot\frac{\overrightarrow{B^{\prime}_{\alpha}D^{\prime}_{\alpha}}}{|B^{\prime}_{\alpha}A{\prime}|}$\\
Le dernier point $D$ = $A - B$ = $C + \overrightarrow{BB^{\prime}_{\alpha}}$

\begin{algorithm}
	$\mathcal{P} \gets \text{Ensemble de points}$\;
	$\mathrm{E} \gets \text{Jarvis}(\mathcal{P})$\;
	$u \gets 1$\; $l \gets 1$\; $r \gets 1$\; $minAire \gets \infty$\;
	\For{$\textbf{each}\thinspace \text{arête} \thinspace \overrightarrow{E_{i+1}E_{i}},\thinspace 0\leqslant i < |E|$}{
		$A^{\prime} \gets E_{i}$\;
		$\alpha \gets \overrightarrow{E_{i+1}E_{i}}$\;
		\While{$|\alpha\times\overrightarrow{E_{u}E_{i}}| \leqslant |\alpha\times\overrightarrow{E_{u+1}E_{i}}|$}{
			$u \gets (u + 1)$ mod $|E|$\;
		}
		$C^{\prime} \gets E_{u}$\;
		\While{$|\alpha\cdot\overrightarrow{E_{r}E_{i}}| \leqslant |\alpha\cdot\overrightarrow{E_{r+1}E_{i}}|$}{
			$r \gets (r + 1)$ mod $|E|$\;
		}
		$B^{\prime} \gets E_{r}$\;
		\If{$i == 0$}{$l = r$\;}
		
		\While{$|\alpha\cdot\overrightarrow{E_{l}E_{i}}| \leqslant |\alpha\cdot\overrightarrow{E_{l+1}E_{i}}|$}{
			$l \gets (l + 1)$ mod $|E|$\;
		}
		$D^{\prime} \gets E_{l}$\;
		$\overrightarrow{B^{\prime}_{\alpha}A^{\prime}} \gets \thinspace$ $\frac{\alpha\cdot\overrightarrow{B^{\prime}A^{\prime}}}{|\alpha|}$\;
		
		$\overrightarrow{D^{\prime}_{\alpha}} \gets \thinspace$ $\frac{\alpha\cdot\overrightarrow{D^{\prime}A^{\prime}}}{|\alpha|}$\;
		
		$height \gets |C^{\prime}_{\alpha}C^{\prime}| \gets \thinspace$ $\frac{\alpha\times\overrightarrow{C^{\prime}A^{\prime}}}{|\alpha|}$\;
		
		$width \gets \overrightarrow{B^{\prime}_{\alpha}A^{\prime}} - \overrightarrow{D^{\prime}_{\alpha}A^{\prime}}$\;
		
		$aire \gets width\times height$\;
		\If{$aire < minAire$}{
			$minAire \gets aire$\;
			$A \gets A^{\prime} - \alpha\cdot \frac{\overrightarrow{B^{\prime}_{\alpha}A^{\prime}}}{|\alpha|}$\;
			$B \gets B^{\prime}_{\alpha} +  \overrightarrow{B^{\prime}_{\alpha}B^{\prime}}\cdot\frac {\overrightarrow{C^{\prime}_{\alpha}C^{\prime}}}{|\overrightarrow{B^{\prime}_{\alpha}B^{\prime}}|}$\;
			$C \gets B + \overrightarrow{B^{\prime}_{\alpha}A^{\prime}}\cdot\frac{\overrightarrow{B^{\prime}_{\alpha}D^{\prime}_{\alpha}}}{|B^{\prime}_{\alpha}A{\prime}|}$\;
			$D \gets C + (A - B)$\;
		}	
	}
	\KwRet{$A, B, C, D$}\;
	\caption{pseudocode de l'algorithm de Toussaint}
\end{algorithm}

\section{Résultat du test}
Le graphe dessous est le résultat de qualité associé à chaque teste de VAROUMAS.\\
\includegraphics[width=\textwidth]{quality_graph.png}\\
On sait que l'aire du rectangle minimum convexe est toujours plus grand que l'aire de polygone convexe.


\section{Performance de l'algorithme}


\end{document}